\documentclass{beamer}
\usetheme{metropolis}           % Use metropolis theme
\title{Vision-based Autonomous Landing in Catastrophe-Struck Environments}
\subtitle{\textit{Authors:} Mayank Mittal, Abhinav Valada, Wofram Burgard}
\date{\today}
\author{Michele Cipriano}
\institute{Control Problems in Robotics: Modeling and control of
    multi-rotor UAVs\\Department of Computer, Control and Management
    Engineering\\Sapienza University of Rome}

% Fontsize of figure smaller than normalsize:
%\setbeamerfont{caption}{size=\scriptsize}

\begin{document}
\nocite{*}

    \maketitle

    \begin{frame}{Introduction}
        \begin{columns}[c,onlytextwidth]
            \column{0.5\textwidth}
                \vspace*{\fill}
                Intro.
                \begin{itemize}
                    \item Why it's important (bioradiolocation, also emergency landing)
                    \item previous work
                \end{itemize}
                \vspace*{\fill}

            \column{0.5\textwidth}
                \begin{figure}
                    \includegraphics[width=\textwidth]{images/Fig1.png}
                \end{figure}
        \end{columns}
    \end{frame}

    \begin{frame}{Technical Approach}
        Brief introduction to ch. 3, describe approach and then discuss details.
        \begin{itemize}
            \item State Estimation
            \item Landing Site Detection
            \item 3D Volumetric Mapping
            \item Landing Trajectory Estimation
        \end{itemize}
    \end{frame}

    \begin{frame}{Technical Approach: State Estimation}
        ORB-SLAM2 + IMU + barometer + GPS (EKF multi-sensor fusion).
    \end{frame}

    \begin{frame}{Technical Approach: Landing Site Detection}
        Cost functions + clustering.
        \begin{equation*}
            J = c_1 J_{DE} + c_2 J_{FL} + c_3 J_{N} + c_4 J_{EC}
        \end{equation*}
        with $c_i \in [0, 1]$ weighting parameters such that
        $\sum_i c_i = 1$.
    \end{frame}

    \begin{frame}{Technical Approach: Landing Site Detection}
        Confidence in \textbf{\textsc{Depth Information}} $J_{DE}$:
        \begin{equation*}
            J_{DE}(p) = 1 - \frac{D(p)^2 - min\{D^2\}}{max\{D^2\}}
        \end{equation*}
    \end{frame}

    \begin{frame}{Technical Approach: Landing Site Detection}
        \textbf{\textsc{Flatness Information}} $J_{FL}$:
        \begin{gather*}
            di(B, p) = min\Big\{\|p-q\| \Big| B(q)=1\Big\} \\
            J_{FL}(p) = di(Canny(D), p)
        \end{gather*}
    \end{frame}

    \begin{frame}{Technical Approach: Landing Site Detection}
        \textbf{\textsc{Steepness Information}} $J_{N}$:
        \begin{gather*}
            \theta = cos^{-1}(\hat{n}_z) \\
            n(p) = exp\left\{ -\frac{\theta^2}{2\theta^2_{th}} \right\}
        \end{gather*}
        description $\theta_{th}=\pi/12$.
    \end{frame}

    \begin{frame}{Technical Approach: Landing Site Detection}
        \textbf{\textsc{Energy Consumption Information}} $J_{EC}$:
        \begin{equation*}
            J_{EC}(p) = \int_{t_0}^{t_f} P(t)dt
        \end{equation*}
    \end{frame}

    \begin{frame}{Technical Approach: 3D Volumetric Mapping}
        OctoMaps and Voxblox
    \end{frame}

    \begin{frame}{Technical Approach: Landing Trajectory Estimation}
        \begin{columns}[c,onlytextwidth]
            \column{0.5\textwidth}
                RRT* and minimum-jerk trajectory.
                \begin{itemize}
                    \item First.
                    \item Second.
                \end{itemize}

            \column{0.5\textwidth}
                \begin{figure}
                    \centering
                    \includegraphics[width=\textwidth]{images/Fig4.png}
                \end{figure}
        \end{columns}
    \end{frame}

    \begin{frame}{Experimental Evaluation}
        Description:
        \begin{itemize}
            \item Hyperrealistic Simulation
            \item Real-World Outdoor (Training Center for Rescue, Germany)
        \end{itemize}
        + computation costs
    \end{frame}

    \begin{frame}{Experimental Evaluation: Hyperrealistic Simulation}
        Hyperrealistic Simulation Experiments
    \end{frame}

    \begin{frame}{Experimental Evaluation: Real-World Outdoor}
        Real-World Outdoor Experiments
    \end{frame}

    \begin{frame}{Experimental Evaluation: Computation Costs}
        \begin{columns}[c,onlytextwidth]
            \column{0.5\textwidth}
                Computation costs

            \column{0.5\textwidth}
                \vspace{0.4cm}
                \begin{figure}
                    \centering
                    \includegraphics[width=\textwidth]{images/Fig8a.png}
                \end{figure}
                \vspace{-0.8cm}
                \begin{figure}
                    \centering
                    \includegraphics[width=\textwidth]{images/Fig8b.png}
                \end{figure}
        \end{columns}
    \end{frame}

    \begin{frame}{Conclusion}
        Conclusion.
    \end{frame}

    \begin{frame}[standout]
        Q\&A
    \end{frame}

    \appendix

    \begin{frame}{References}
        \bibliography{bibliography}
        \bibliographystyle{ieeetr}
    \end{frame}

\end{document}
