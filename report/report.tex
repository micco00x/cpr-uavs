%++++++++++++++++++++++++++++++++++++++++
% Don't modify this section unless you know what you're doing!
%\documentclass[letterpaper,12pt]{article}
\documentclass[a4paper]{article}
\usepackage{tabularx} % extra features for tabular environment
\usepackage{amsmath}  % improve math presentation
\usepackage{graphicx} % takes care of graphic including machinery
%\usepackage[margin=1in,letterpaper]{geometry} % decreases margins
\usepackage{cite} % takes care of citations
%\usepackage[final]{hyperref} % adds hyper links inside the generated pdf file
%\hypersetup{
%	colorlinks=true,       % false: boxed links; true: colored links
%	linkcolor=blue,        % color of internal links
%	citecolor=blue,        % color of links to bibliography
%	filecolor=magenta,     % color of file links
%	urlcolor=blue
%}
%++++++++++++++++++++++++++++++++++++++++
\usepackage{indentfirst}
\usepackage{tensor}
\usepackage{amssymb}
\allowdisplaybreaks
\usepackage{bm}
\newcommand{\at}[2][]{#1|_{#2}}
\newcommand\numberthis{\addtocounter{equation}{1}\tag{\theequation}}
\newcommand\norm[1]{\left\lVert#1\right\rVert}

\usepackage{hyperref}

\usepackage[none]{hyphenat} % Avoids to go out of margin

\begin{document}
\sloppy

\title{Modeling and Control of Multi-Rotor UAVs\\Paper Proposal}
\author{Michele Cipriano}
\date{\today}
\maketitle

%\begin{abstract}
%In this experiment we studied a very important physical effect by measuring the
%dependence of a quantity $V$ of the quantity $X$ for two different sample
%temperatures.  Our experimental measurements confirmed the quadratic dependence
%$V = kX^2$ predicted by Someone's first law. The value of the %mystery parameter
%$k = 15.4\pm 0.5$~s was extracted from the fit. This value is
%not consistent with the theoretically predicted $k_{theory}=17.34$~s. We attribute this
%discrepancy to low efficiency of our $V$-detector.
%\end{abstract}


\section{Introduction}
The papers I chose to propose cover different aspects of UAVs research.
The first one, ``Real-Time Planning with Multi-Fidelity Models for Agile Flights
in Unknown Environments'' \cite{DBLP:journals/corr/abs-1810-01035}, introduces
a real time planning algorithm for UAVs in unknown environments, developing
three different models of the UAV in order to guarantee computational
tractability. The second one, ``Vision-based Autonomous Landing in
Catastrophe-Struck Environments'' \cite{DBLP:journals/corr/abs-1809-05700},
is a practical use case of UAVs and it introduces a framework that allows
a drone to land in a catastrophic environment, creating in the meanwhile
a precise reconstrution of the world in order to make a human operator
perform more complex analysis. The last paper,
``Neural Lander: Stable Drone Landing Control using Learned Dynamics''
\cite{DBLP:journals/corr/abs-1811-08027}, introduces a nonlinear controller
that models the ground effects caused by the interaction between
multi-rotor airflow and the environment with a DNN, formally proving its
stability and presenting results on a real quadrotor.

\section{Real-Time Planning with Multi-Fidelity Models for Agile Flights
    in Unknown Environments}
This paper \cite{DBLP:journals/corr/abs-1810-01035} introduces a real time
planning algorithm for UAVs in unknown environments.

The approach is to develop
three multi-fidelity models of the UAV in order to make sure the hierarchical
planning architecture (composed by a global and a local planner)
is behaving correctly. In particular, the Jump Point Search (JPS) algorithm has been
used as a global planner to find the shortest path to the goal from the
current position of the UAV, guaranteeing completeness and optimality. The local
planner monitors the JPS solution looking for changes between each replan. If
a change is detected, a new trajectory composed by a high (jerk-controlled),
a medium (velocity-controlled) and a low fidelity
model (JPS) is computed.

The environment is represented using a sliding map in order to make it possible to
compute collision check for each of the three fidelity models. In particular,
a depth map is obtained from the UAV and fused into an occupancy grid using a
k-d tree data structure, ensuring that the local planner is using the most
recent data of the map.

The experiments have been performed in simulation considering
a random cluttered forest, a bugtrap and a cluttered office scenario,
making comparison with other methods. The experiments on hardware have used
a UAV with all the perception, planning and control algorithms running
onboard.

This method guarantees computational tractability keeping
replanning times in the order of 5-40 ms in simulation.

\textbf{Personal Comments:} I have found this paper interesting because of
the state of the art approach in real time planning when dealing with
unknown environments. In particular, I liked the idea of considering
multi-fidelity models to make sure the UAV is behaving correctly, always
achieving the goal while keeping the computation real-time. The authors
managed to describe in detail their approach, considering multiple
scenarios and comparing their methods with several different algorithms.
A YouTube video with some of the experiments is available at
\url{https://www.youtube.com/watch?v=XVaT_c8uSTA}.

\section{Vision-based Autonomous Landing in Catastrophe-Struck Environments}
This paper \cite{DBLP:journals/corr/abs-1809-05700} introduces an
autonomous landing method based on vision algorithms when dealing with
catastrophic environments.

The approach aims at making a UAV equipped with a bioradar to autonomously
land on debris piles in order to locate the survivors in a natural
disaster. In particular, the authors developed a multi-sensor fusion method to
correctly estimate the pose of the UAV in two different 3D representations
of the environment. The first one, which uses Octomap, is used as an internal
map to make trajectory planning computationally tractable. The second one,
which uses Voxblox, is a mesh reconstruction which is transmitted to
the ground station in order to be analyzed by a human operator. In this way
the UAV will be able to do both planned and emergency landings.

The landing site is evaluated in three steps. Initially a costmap is computed
considering the terrain flatness, steepness, the depth accuracy and the
energy consumption. Then, a set of candidates is chosen from the costmap. In the
end, a clustering algorithm is used to reduce the number of unique landing sites.
The landing site is chosen by solving an unconstrained nonlinear optimization
problem and a minimum-jerk trajectory is computed using a variant of RRT*.

The experiments have been performed on a virtual environment, which simulates
a small city, and in Training Center of Rescue (Germany), where a real quadrotor
has been used.

This method, which makes use of a backend to use Octomap and Voxblox, makes it
possible to compute the entire landing site detection algorithm in 167.5 ms,
allowing it to be used on a real UAV.

\textbf{Personal Comments:} I have found this paper interesting because of
its practical application. In particular, I liked how the authors
decided to structure the pipeline of their algorithm so that it is possible
to make it run on a real UAV. The idea of using two different 3D maps
not only makes the UAV able to take decisions on its own, but also allows a
human expert to understand the real situtation of the environment, enabling
more complex decisions and reducing the gap between research and real-world
applications.

\section{Neural Lander: Stable Drone Landing Control using Learned Dynamics}
This paper \cite{DBLP:journals/corr/abs-1811-08027} introduces a deep learning
based nonlinear controller that controls the quadrotor landing, modeling
the ground effects caused by the interaction between multi-rotor airflow and
the environment.

Considering the dynamics of the quadrotor:
\begin{align}
    m\bm{\dot{v}} &= m\bm{g} + R\bm{f}_u + \bm{f}_a \\
    J\bm{\dot{\omega}} &= J\bm{\omega} \times \bm{\omega} + \bm{\tau}_u + \bm{\tau}_a
\end{align}

\noindent with $\bm{\dot{p}} = \bm{v}$, $\bm{p} \in \mathbb{R}^3$ global
position, $\bm{v} \in \mathbb{R}^3$ linear velocity, $\bm{R} \in SO(3)$
attitude rotation matrix, $\bm{\omega} \in \mathbb{R}^3$ body angular velocity,
$\bm{g} = [0, 0, -g]^T$ gravity vector, $\bm{f}_u = [0, 0, T]^T$ total thrust
and $\bm{\tau}_u = [\tau_x, \tau_y, \tau_z]^T$ body torques, the approach aims
at developing a DNN that models the unknown disturbances forces $\bm{f}_a$
and torques $\bm{\tau}_a$ which originates from complex aerodynamics
interactions between the quadrotor and the environment. In particular,
since the $\bm{\tau}_a$ is bounded during landing and take-off, only
$\bm{f}_a$ needs to be studied.

The authors implemented a DNN with spectral normalization to guarantee the
stability of the output. In particular, they exploited this kind on
normalization to formally prove that the nonlinear controller used to
determine the force exerted by the rotors
is stable. The proof is
built upon the assumption that $\bm{p}_d(t)$, $\bm{\dot{p}}_d(t)$ and
$\bm{\ddot{p}}_d(t)$ are bounded, the inputs $\bm{u}$ of the system update
much faster than the position controller and the approximation error on
$\bm{\hat{f}}_a$ is upper bounded.

The experiments have been performed using the Intel Aero quadrotor. First,
bench tests have been done in order to estimate the mass, the diameter of
the rotors, the thrust coefficient, the air density and the
gravity. Then, data has been collected by making the UAV perform some
maneuvers in order to create a dataset. In the end, a DNN has been trained and
the resulting model has been compared to a PD controller for take-off and
landing tasks in both 1D and 3D.

This method allows the drone to precisely land on the ground surface in both
1D and 3D cases, reducing drifts in $x$, $y$ directions and learning about
non-dominant aerodynamics such as air drag. It is also interesting to notice
that the lack of spectral normalization can even result in crashes, meaning
that it is important to have a rigorous theoretical analysis to guarantee
the stability of the controller used on the drone.

\textbf{Personal Comments:} I have found this paper interesting because of
its theoretical approach to the field. Using a neural network as it is often
results in undesired behaviours, going in the opposite direction of what it
needs to be done in robotics. The authors managed to properly model the
problem, formally proving the stability of their new approach, allowing it
to be used on a real UAV and introducing a new research approach that
integrates learning methods with control theory.
A YouTube video with some of the experiments is available at
\url{https://www.youtube.com/watch?v=C_K8MkC_SSQ}.

%++++++++++++++++++++++++++++++++++++++++
% References section will be created automatically
% with inclusion of "thebibliography" environment
% as it shown below. See text starting with line
% \begin{thebibliography}{99}
% Note: with this approach it is YOUR responsibility to put them in order
% of appearance.

\clearpage
\bibliography{bibliography}
\bibliographystyle{ieeetr}

%\begin{thebibliography}{99}

%\bibitem{melissinos}
%A.~C. Melissinos and J. Napolitano, \textit{Experiments in Modern Physics},
%(Academic Press, New York, 2003).

%\bibitem{Cyr}
%N.\ Cyr, M.\ T$\hat{e}$tu, and M.\ Breton,
% "All-optical microwave frequency standard: a proposal,"
%IEEE Trans.\ Instrum.\ Meas.\ \textbf{42}, 640 (1993).

%\bibitem{Wiki} \emph{Expected value},  available at
%\texttt{http://en.wikipedia.org/wiki/Expected\_value}.

%\end{thebibliography}


\end{document}
